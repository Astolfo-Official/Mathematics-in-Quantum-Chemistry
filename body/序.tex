化学教育今年来的发展趋势,迫使学化学的学生在学习进程中接触量子力学的应用越来越早。
本科化学课程表的这一发展方向之所以重要其原因如下:
物理-化学信息的宝库来源于建立在量子力学的基础上的分子光谱和其他技术,表述化学的发展趋势是建立在量子力学提供的概念的基础上;
理论化学的发展方向是用量子力学充分解释分子现象。

将同学骤然引到这种艰难和令人激动的学习中,常使他们不能适应表达这些内容的数学公式,把握不住已经习以为常的牛顿世界和分子现象世界之间的联系,不能将新建立的理论知识应用到分子结构和分子运动问题上。

在本书中我们只准备讲数学中两个重要课题和物理学中两个重要课题。
它们是数学中的正交函数微积分和向量空间代数,物理学中的拉格朗日经典力学和哈密顿经典力学及它们在分子运动中的应用。
我选择这四个课题是因为它们与现代量子化学密切相关,特别与量子力学在分子光谱中应用密切相关。
对分子光谱的强调反映了我个人对这个成长着的和普及的领域的兴趣和工作;它也使我从这本小书中去掉了对我的同时和他们的学生感兴趣的课题。
我们去掉了小定理,电学,磁学和辐射物理,因为它们在别的书中一般讲比本书处理得更好,并且容许做更深入的讨论;由于同样原因,将群论与微分方程包括近似解法也留给别的书来讲。

本书试图一般地为量子化学,特别为分子光谱准备物理和化学基础。
读者应有包括偏微分和重积分在内的微积分(通常约为一年半),一年物理和一年化学的准备知识。
这个材料曾用做Bryn Mawr学院,为具有上述准备知识的同学开设的一学期课程“化学家用的应用数学”的基础;在此课程后接着学习初等量子力学。

作者对Addison Wesley出版公司允许从其出版物上取材和W. Z. Benjamin公司的不断帮助和鼓励表示感谢。
\begin{flushright}
    Jay \ Martin \ Anderson \\
    Bryn \ Mawr Pennsylrania \\
    1965年10月
\end{flushright}