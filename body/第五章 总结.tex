本书的前四章是为量子力学,特别是为分子光谱学准备舞台而设计的。
为了完成舞台道具的准备(事实上是将幕稍启)我们要触及两点。
第一,经典力学和量子力学之间的联系,第二,波动力学观点和矩阵力学观点的综合。

\section{经典力学和量子力学之间的桥}
欲严格地介绍量子力学,应该先知道在十九世纪末和二十世纪初那些令人激动的实验,这些实验要求的理论说明,不是来自牛顿力学和麦克斯威电磁学。
重温此背景将使我们离开本书的目的。因此,本节限定在基本理论的水平上比较经典力学和量子力学,至于量子力学的实验基础则留给读者到别处去欣赏。

按传统的讲法,量子力学课是从评述经典力学的失败开始。对量子力学的最常用的处理方法,是用波动-微粒二象性这个十分自然的观点,并以德布罗意的工作为例。
在开始时就应该认识到,习惯上将经典波动方程和德布罗意的动量波长关系式结合,并不是薛定谔方程的推导,而薛定谔方程则是量子力学发展的基本方程。
只能将薛定谔方程做为假定,从这个假定出发形成对物理世界的看法。此看法只借助相对论和场论的导引在描述亚微观物质上取得巨大的成就。无论如何,波动力学不能推导而只是假定。
量子力学的另一种处理方法,更抽象和更基本些,是建立算符和对应的力学量之间的联系。沿此方向就能提出量子力学的简单的和自洽的一组假定,这些假定是抽象的和基本的。
再一次提出,在开始时就不应把这里讲的处理方法理解为推导,只能理解为假定。它比薛定谔力学的物理内容既不多也不少。作者希望, 用第三章和第四章的术语陈述量子力学而不用人们可能不太熟悉的波动术语,
至少在比较量子力学和经典力学时,会产生新的认识,甚至可能使在两种物理学之间的峡沟上搭桥,不是信念的飞跃而是推理的发展。

玻尔认为,由于经典力学对宏观体系是“正确的”,因此量子力学的引入不能全盘根除过去的经典理论。量子力学必须把经典力学做为极限情况包括在内。
这种陈述木身可以看做是对应原理。怎样用公式表示出由量子力学得到经典力学的极限的本性呢?这个问题能在许多复杂的水平上回答。
这里只给一种指出量子力学的经典极限的方法以及它们之间的对应性;还有别的表示这种对应性的方法。

例如,按$Dicke$和$Wittke$的处理,我们可从量子力学的下述假定开始:每个物理可测量结果是对应该量的算符的本征值。

在经典力学极限中,物理量的算符特性失去了;或更精确讲,物理量$Q$的算符$\mathscr{Q}$必须用“$Q$乘被作用函数”来定义,即$\mathscr{Q}f \rightarrow Qf$。
若这一点成立,那么在经典力学极限中对应于物理的全部算符是可对易的。在唯有量子力学才能“正确地”描述问题的情况下,全部算符是不可对易的。因此, 量子力学的经典力学极限的本性必定是
\[[\mathscr{A},\mathscr{B}]=\text{(小的数)(对应于A,B的某函数的算符)} \tag{5-1}\]
在发展最子理论的过程中,一个“小的数”一次又一次地出现。这个数是普朗克常数。事实上,方程5-1中的数就是$\hbar$这在理论最终与实验比较中得到证实。现在
\[[\mathscr{A},\mathscr{B}]=i\hbar\text{(对应于A,B的某函数的算符)} \tag{5-2}\]
我们对经典力学和量子力学之间的对应性的寻找,差不多完成了。由于量纲的原因,“A,B的函数”的量纲,必须是($AB$/尔格-秒)因为$i\hbar$的单位是尔格-秒。这样的函数就是$A$和$B$的泊松括号因此,
\[[\mathscr{A},\mathscr{B}]=i\hbar\text{(对应于$\{A,B\}$的算符)} \tag{5-3}\]
它就是$Dicke$和$Wittke$的第七个假定。方程5-3不是推导出来的,而是“规定”或“辅助导引”出来的。
从经典力学和量子力学之间的对应性和极限的陈述出发,我们可以推导(现在用推导这个词是正确的)出一些重要的结果。

第一,比较力学量和哈密顿函数的泊松括号,与对应的算符和哈密顿算符的换位子。
\[\dv{f}{t}=\{f,H\}+\frac{\partial f}{\partial t} \tag{4-48}\]
再求助于经典力学极限的原理,算符$\mathscr{F}$的行为就像函数$f$一样,
\[\dv{\mathscr{F}}{t}=-\frac{i}{\hbar}[\mathscr{F},\mathscr{H}]+\frac{\partial \mathscr{F}}{\partial t}=\frac{i}{\hbar}[\mathscr{H},\mathscr{F}]+\frac{\partial \mathscr{F}}{\partial t} \tag{5-4}\]
这就是海森堡运动方程。海森堡的观点(重点放在线性算符及其矩阵表示),将力学量的全部时间变量通过方程5-4放到其对应的算符中。
此方程使我们能用守恒定理来表述对应原理:在经典力学中运动守恒的量,在量子力学中仍然是运动守恒的量。
例子\footnote{原文为“特例”,我觉得是翻译的问题,原作者可能想表达强调。}是,若经典哈密顿量不显函时间,则能量是守恒的:
\[\dv{H}{t}=\{H,H\}+\frac{\partial H}{\partial t}=\{H,H\}=0 \tag{5-4a}\]
其量子力学的类似关系为
\[\dv{\mathscr{H}}{t}=\frac{i}{\hbar}[\mathscr{H},\mathscr{H}]+\frac{\partial \mathscr{H}}{\partial t}=\frac{i}{\hbar}[\mathscr{H},\mathscr{H}]=0 \tag{5-4b}\]

薛定谔用的另一种观点
\footnote{原书的脚注:这里使用“观点”一词表示运动方程在量子力学中是用什么方式来描述,别的作者用“表示”表示此概念如“海森堡表示”薛定得表示”。但我们用“表示”表示十分不同的意思,因而这里不用它。}
,是使算符不显函时间,而其本征值显函时间。为此,开始时用其本征函数为$\phi$的算符的本征值$f$的运动方程。
\[\dv{\mathscr{F}}{t}=\dv{t}\bra*{\phi}\mathscr{F}\ket*{\phi}=\bra*{\dv{\phi}{t}}\mathscr{F}\ket*{\phi}+\bra*{\phi}\mathscr{F}\ket*{\dv{\phi}{t}} \tag{5-6}\]
由方程5-4得
\[\dv{\mathscr{F}}{t}=\bra*{\phi}\frac{i}{\hbar}[\mathscr{H},\mathscr{F}]\ket*{\phi}=\bra*{\phi}\frac{i}{\hbar}\mathscr{H}\mathscr{F}-\frac{i}{\hbar}\mathscr{F}\mathscr{H}\ket*{\phi}\]
\[=\bra*{-\frac{i}{\hbar}\mathscr{H}\phi}\mathscr{F}\ket*{\phi}+\bra*{\phi}\mathscr{F}\ket*{-\frac{i}{\hbar}\mathscr{H}\phi} \tag{5-7}\]
为使方程5-6和方程5-7一致,则要求
\[\mathscr{H}\phi=i\hbar\frac{\partial \phi}{\partial t} \tag{5-8}\]
这就是薛定谔方程。

这样,我们就看到了经典力学和量子力学之间的简洁的对应关系能够建立起来,它既可引出海森堡运动方程又能引出薛定谔运动方程,并伴随着守恒定理。

\section{矩阵力学和波动力学的综合}
我们已经讲了本征值-本征向量问题的两种观点。在第一种观点里,算符是微分算符,因而本征值方程是微分方程。
在第二种观点里,算符用矩阵表示,因而本征值方程是用一组联立线性齐次方程表示。这两种观点的综合可在希伯特空间的概念中找到。

我们需要证明,第三章的代数法仍可应用于连续变量问题(许多量子力学问题所具有的)。
若定义希伯特空间为,由有定义内积(方程3-1)的归一化函数张成的向量空间,那我们就可以建立矩阵力学和波动力学的综合。
希伯特空间的概念允许向量空间是无限维的。此无限个数可以是分立的,并且可如基组$\{H_n(x)\}$中指标$n$那样标记为$n=0,1,2\cdots$。或此无限个数可以是连续的,如直线上的点那样不能用数标记。

对希伯特空间作进一步讨论使我们离题太远;作者只希望指出特征函教和特征向量之间的相似性质。
在希伯特空间里,算符的相阵表示可以是无限维的;本征函数可用无限长的列向量表示。
例如,位置算符$\mathscr{X}$具有连续域,或特征值谱,如$\{x'\}:\mathscr{X}\ket*{x'}=x'\ket*{x'}$。
希伯特空间中的任何本征向量$\ket*{\xi'}$都可用$\mathscr{X}$的本征向量(因为它们形成一完备正交归一组)展开。
\[\ket*{\xi}=\sum\ket*{x'}\bra*{x'}\ket*{\xi}=\int\ket*{x'}\bra*{x'}\ket*{\xi}\dd{x} \tag{5-9}\]
因希伯特空间是无限维的,所以我们用对连续变量$x'$积分代替求和。系数$\bra*{x'}\ket*{\xi}$是(复)函数,它组成以$\ket*{x'}$为基的向量$\ket*{\xi}$的表示。
我们可称此复函数集合为波函数:$\Psi_{\xi}(x')=\bra*{x'}\ket*{\xi}$。二希伯特空间向量的内积$\bra*{\xi}\ket*{\eta}$给出与第二章的理论的最终联系:
\[\bra*{\xi}\ket*{\eta}=\int \dd{x}'\int \dd{x}''\bra*{\xi}\ket*{x'}\bra*{x'}\ket*{x''}\bra*{x''}\ket*{\eta}=\int \dd{x}'\bra*{\xi}\ket*{x'}\bra*{x'}\ket*{\eta}=\int \dd{x}'\Psi_{\xi}^*(x')\Psi_{\eta}(x') \tag{5-10}\]
它与方程2-1相同\footnote{原文5-10式最后一个等号后面的$\Psi_{\eta}(x')$误写为$\Psi^*_{\eta}(x')$(笑}。

到此,我们结束了对与量子化学有关的数学和物理的突击。作者的希望,是同学们在量子力学的原理和应用的过程中,不要被同时紧张地领会表达量子力学的数学语言所羁绊。

\begin{problemset}
\item 证明方程5-3的量纲是正确的。
\item 应用方程5-3所表达的对应原理,求笛卡儿坐标和笛卡儿直线动量之间的对易关系。
\end{problemset}